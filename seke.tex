%!TEX root=seke.tex
% mainfile: seke.tex

\documentclass[10pt,twocolumn]{article}

\usepackage[table]{xcolor}% http://ctan.org/pkg/xcolor
\usepackage{latex8}
\usepackage{times}
\usepackage{inconsolata}
\usepackage{pifont}

\usepackage{graphicx}
\usepackage{algorithm}
\usepackage{algorithmic}

\usepackage{tikz}
\usetikzlibrary{shapes,arrows,shadows}
\usepackage{amsmath,bm,times}
\usepackage{verbatim}

\usepackage{subcaption}
\usepackage{bibspacing}

\usepackage[small,compact]{titlesec}
% \usepackage{titlesec}

\newcommand{\goallegheny}{$^{\mbox{\footnotesize \ding{72}}}$}
\newcommand{\gosheffield}{$^{\mbox{\footnotesize \ding{73}}}$}
\newcommand{\gospace}{$\;$}

% \makeatletter
% \renewenvironment{thebibliography}[1]{%
%   \vspace*{-.1in}
%   \@xp\section\@xp*\@xp{\refname}%
%   % \normalfont\footnotesize\labelsep .5em\relax
%   \renewcommand\theenumiv{\arabic{enumiv}}\let\p@enumiv\@empty
%   \vspace*{-.1in}% NEW
%   \list{\@biblabel{\theenumiv}}{\settowidth\labelwidth{\@biblabel{#1}}%
%     \leftmargin\labelwidth \advance\leftmargin\labelsep
%     \usecounter{enumiv}}%
%   % \sloppy \clubpenalty\@M \widowpenalty\clubpenalty
%   % \sfcode`\.=\@m
% }{%
%   \def\@noitemerr{\@latex@warning{Empty `thebibliography' environment}}%
%   \endlist
% }
% \makeatother

\begin{document}

% \title{Empirically Evaluating the Efficiency of Search-based \\ Test Data
% Generation for Relational Database Schemas\vspace*{-.1in}}

\title{\vspace*{-.6in}Automatically Evaluating the Efficiency of \\ Search-Based Test Data
Generation for Relational Database Schemas\vspace*{-.1in}}

\author{Cody Kinneer \goallegheny                \and
        Gregory M.\ Kapfhammer \goallegheny      \and
        Chris Wright \gosheffield                \and
        Phil McMinn \gosheffield \vspace*{-.1in}
      }

\affiliation{
      \goallegheny \gospace Allegheny College       \and
      \gosheffield \gospace University of Sheffield
}

\maketitle

\begin{abstract}

% When evaluating an algorithm, it is often useful to speak of it's efficiency in terms of it's worst-case complexity.
% This is the case for search-based test data generation tools.
% on the search-based data generation tool \textit{SchemaAnalyst}.

% This paper introduces a framework for conducting automated empirical studies of algorithms by doubling
% the size of the input and observing the change in runtime.

% After describing a way to systematically doubling the size of structured data, we report on a study demonstrating the
% presented method's effectiveness.

The characterization of an algorithm's worst-case time complexity is useful because it succinctly captures how its
runtime will grow as the input size becomes arbitrarily large.  However, for certain algorithms---such as those
performing search-based test data generation---a theoretical analysis to determine worst-case complexity is difficult to
generalize and thus not often reported in the literature.  This paper introduces a framework that empirically determines
an algorithm's worst-case time complexity by doubling the size of the input and observing the change in runtime.  Since
the relational database is a centerpiece of modern software and the database's schema is frequently untested, we apply
the doubling technique to the domain of data generation for relational database schemas, a field where worst-case time
complexities are unknown.  In addition to demonstrating the feasibility of suggesting the worst-case runtimes of the
chosen algorithms and configurations, the results of our study reveal performance trade-offs in schema testing
strategies.

\end{abstract}

%!TEX root=seke.tex
% mainfile: ../seke.tex

% This content is duplicated from the conference paper to which this tool paper corresponds

{\bf Introduction to doubling}. A useful understanding of an algorithm's efficiency, the worst-case time complexity
gives an upper bound on how an increase in the size of the input, denoted $n$, increases the execution time of the
algorithm, or $f(n)$.  This relationship is often expressed in the ``big-Oh'' notation, where $f(n)$ is $O(g(n))$
means that the time increases by no more than on order of $g(n)$. Since the worst-case complexity of an algorithm is
evident when $n$ is large~\cite{mcgeoch2012}, one approach for determining the big-Oh complexity of an algorithm is to
conduct a doubling experiment with increasingly bigger input sizes. By measuring the time needed to run the algorithm
on inputs of size $n$ and $2n$, the algorithm's order of growth can be determined~\cite{mcgeoch2012}.

The goal of a doubling experiment is to draw a conclusion regarding the efficiency of the algorithm from the ratio
$f(2n)/f(n)$ that represents the factor of change in runtime from inputs
of size $n$ and $2n$. For instance, a ratio of $2$
would indicate that doubling the input size resulted in the runtime's doubling, leading to the conclusion that the
algorithm under study is $O(n)$ or $O(n\log n)$.
Table~\ref{table:ratios} shows some common time complexities and
corresponding ratios.

\begin{table}[h!]
  \vspace*{-.05in}
  \begin{center}
    \begin{tabular}{c|l}
      Ratio $f(2n)/f(n)$ & Worst-Case Conclusion              \\ \hline
      1                  & constant or logarithmic \\
      2                  & linear or linearithmic  \\
      4                  & quadratic               \\
      8                  & cubic                   \\
      % x                & $O(n^{\log x})$
    \end{tabular}
  \end{center}
  \vspace*{-.225in}

  \caption{Conclusions for worst-case time complexity.}~\label{table:ratios}
  \vspace*{-.225in}

\end{table}

\input{writing/background}
\input{writing/technique}
% begin experiment section
\input{writing/experimentdesign}
\input{writing/results_bigOh}
\input{writing/results_trees}
\input{writing/results_bwplots}
\input{writing/results_tables}
\input{writing/threats}
\input{writing/conclusion}

% GMK NOTE: I removed this and moved it into the background section
% \input{writing/relatedworks}

% \titleformat{\chapter}{\huge\bf}{}{10pt}{\arabic{chapter} \ }[\titlerule]
\setlength{\bibitemsep}{.075in}
{\footnotesize
  \bibliographystyle{IEEEtran}
% \vspace*{-.05in}
\bibliography{bibtex/seke}}

\end{document}
