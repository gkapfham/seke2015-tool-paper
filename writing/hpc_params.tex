%!TEX root=seke.tex
% mainfile: ../seke.tex

    % is parallel co

{\bf Deploying on a high-performance cluster}.  Since the performance of
\textit{SchemaAnalyst} may depend on a number of factors (i.e., criterion, data
generator, schema, and doubling strategy) a comprehensive survey of the
parameter space may be conducted by performing a doubling experiment for each
configuration. While computationally expensive, an experiment of this scale is
possible by using a high-performance computing (HPC) cluster. Each doubling
experiment can be run independently on a separate node of the cluster;
\toolname can combine the resulting data for a later analysis. Data mining techniques
can then be leveraged to interpret an algorithm's performance trade-offs.

 {\bf Parameter Tuning}.  While \toolname greatly eases the process of
 conducting doubling experiments, its accuracy and performance is sensitive to
 the settings of the system's parameters.  In particular, the
 $\mathit{tolerance}$ and $\mathit{lookback}$ values can result in a doubling
 experiment terminating prematurely or continuing indefinitely.  To complicate
 the issue further, the parameters must be re-tuned based on hardware
 properties of the machine(s) being used and the performance characteristics of
 the implementation being studied.

  The reliability of the tool and repeatability of its results would be further
  improved if \toolname could select good settings for these parameters
  automatically. A reasonable parameter tuning strategy could be to run
  \toolname on various algorithms of known worst-case time complexities, such
  as the sorting algorithms, and lower the \textit{tolerance} threshold until
  \toolname reliably infers the big-Oh time complexities.

