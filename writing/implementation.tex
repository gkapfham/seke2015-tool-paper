    % TODO fix informal language here
    % the tone I would use when presenting the demo, not sure if
    % appropriate for the tool paper
    {\bf Implementation}.
    \toolname is implemented as an abstract class in the Java
    programming language.  To use \toolname to evaluate your own
    algorithm $A$, you need only extend the \textit{DoublingExperiment}
    class to provide your own $f$ and $d$ functions.  $f$ should be
    implemented by providing a \texttt{double timedTest()} method,
    and $d$ should be implemented by providing a \texttt{void doubleN()}
    method. Note that these methods do not accept any parameters,
    and only \texttt{timedTest()} returns a value. The programmer
    must ensure that \texttt{timedTest()} returns the runtime for the
    current input size, and that when \texttt{doubleN()} is called,
    the input size is doubled.  Actually initializing and storing the
    input should be handled by the specific implementation. The
    \texttt{runExperiment()} method can then be called to to conduct
    a doubling experiment, and the \texttt{printBigOh()} method can
    be called to show the result. Figure~\ref{fig:qsprogram} shows an
    implementation that conducts a doubling experiment on the QuickSort
    algorithm.

    %TODO GMK: do we need to cite the quicksort implementation anywhere?

    % we can cut this, but it does show how easy the tool makes things
    
    \begin{figure}[t]
    \lstinputlisting[basicstyle=\scriptsize]{content/QuickSortTrimmed.java}
    \vspace{-0.15in}
    \caption{A simple Java class that performs a performance evaluation
    on the QuickSort algorithm.}\vspace{-0.20in}
    \label{fig:qsprogram}
    \end{figure}
