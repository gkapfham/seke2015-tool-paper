    % TODO fix informal language here
    % the tone I would use when presenting the demo, not sure if
    % appropriate for the tool paper

    {\bf Implementation}.  \toolname is implemented as a package of
    classes in the Java programming language~\cite{tool}.  To use \toolname to
    evaluate a new algorithm $A$, you only need to extend the
    \texttt{DoublingExperiment} class to provide your own $f$ and $d$
    functions.  The $f$ function should be implemented by providing a
    \texttt{double timedTest()} method, and $d$ should be implemented by
    providing a \texttt{void doubleN()} method. Note that these methods
    do not accept any parameters, and only \texttt{timedTest()} returns
    a value. The programmer must ensure that \texttt{timedTest()}
    returns the runtime for the current input size, and that when
    \texttt{doubleN()} is called, the input size is doubled;
    initializing and storing this input should be handled by the specific
    implementation. The \texttt{runExperiment()} method can be called to
    conduct a doubling experiment and \texttt{printBigOh()} can be
    called to show the result. Figure~\ref{fig:qsprogram} shows an
    complete Java class that conducts a doubling experiment on
    QuickSort; note the simplicity of the implementation when using
    \toolname.

    %TODO GMK: do we need to cite the quicksort implementation anywhere?

    % we can cut this, but it does show how easy the tool makes things

    \begin{figure}[t]
    \lstinputlisting[basicstyle=\scriptsize]{content/QuickSortTrimmed.java}
    \vspace{-0.15in}
    \caption{A simple Java class that performs a performance evaluation
    on the QuickSort algorithm.}\vspace{-0.20in}
    \label{fig:qsprogram}
    \end{figure}
